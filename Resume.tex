\documentclass{persian-resume}

\begin{document}

\begin{center}
  {\Huge \textbf{
    علیرضا رضایی
  }} \\
  \vspace{1.5em}
  Alireza10Rezaei10@email.com \hspace{1cm} 5671 496 0996 \hspace{1cm} 5397 924 0938\\
  \vspace{1em}
  Github.com/Alireza10Rezaei10\\
  \vspace{1em}
    1378/06/18
\end{center}

\section{تحصیلات}
\subsection*{دیپلم ریاضی-فیزیک}
دبیرستان سمپاد بابل\\
رتبه‌ی ۲۲۹ منطقه‌ی ۲ کنکور کارشناسی

\subsection*{
کارشناسی فیزیک
}
دانشگاه صنعتی شریف\\
معدل 82.15\\
رتبه‌ی ۵۲ کنکور کارشناسی ارشد فیزیک

\subsection*{
کارشناسی ارشد فیزیک
}
دانشگاه تهران\\
گرایش اپتیک و لیزر

\section{
    سوابق آموزشی
}
\subsection*{
تدریس‌یار
}
\begin{tabular*}{\textwidth}{@{\extracolsep{\fill}} rl}
کاربردهای لیزر (دکتر رسول صدیقی بنابی) & ترم اول ۱۴۰۲ \\
فیزیک لیزر (دکتر رسول صدیقی بنابی) & ترم دوم ۱۴۰۲  \\
آزمایشگاه لیزر (دکتر رسول صدیقی بنابی) & ترم دوم ۱۴۰۲  \\
فیزیک اپتیک (دکتر سید نادر سید ریحانی) & ترم دوم ۱۴۰۲  \\
\end{tabular*}

\subsection*{
    آموزش‌های جانبی
}

\begin{tabular*}{\textwidth}{@{\extracolsep{\fill}} rl}
برگزاری کارگاه‌های "کاردستی با کد"\\ (حلقه لمبدا، انجمن علمی، دانشکده‌ی فیزیک‌، دانشگاه صنعتی شریف) & ۱۴۰۰\\
برگزاری چندین دوره آموزش خصوصی کدنویسی وب و مبانی برنامه نویسی & ۱۴۰۰ تا کنون\\
برگزاری کارگاه‌های آموزش مقدماتی الکترونیک & ۱۴۰۲
\end{tabular*}

\section{
    سابقه‌های شغلی
}
\subsection*{
آزمایشگاه لیزر
}
دستیار تحقیقاتی آزمایشگاه لیزر دانشکده‌ی فیزیک دانشگاه صنعتی شریف

\subsection*{
استارتاپ بازاریابی دیجیتال "نووا"
}
از بنیان‌گذاران و مدیر فنی

\subsection*{
خانه علم ایران
}
از بنیان‌گذاران سبک جدید خانه علم ایران، مدیر عامل و مدیر فنی

\subsection*{
شرکت ساخت‌وساز رویاسازان
}
سرپرست تیم توسعه‌ی وبسایت

\section{
    سابقه‌های ورزشی و اجرایی
}
\subsection*{
    ورزشی
}
\begin{tabular*}{\textwidth}{@{\extracolsep{\fill}} rl}
کاپیتان تیم فوتبال دانشکده‌ی فیزیک دانشگاه صنعتی شریف در مسابقات استاد-کارمند-دانشجو & ۱۴۰۲\\
پاسور دوم تیم والیبال دانشکده‌ی فیزیک دانشگاه صنعتی شریف در مسابقات دانشجویی & ۱۴۰۲
\end{tabular*}
\subsection*{
    اجرایی
}
\begin{tabular*}{\textwidth}{@{\extracolsep{\fill}} rl}
از مجریان دانشجویی گردهمایی سراسری انجمن فیزیک ایران در دانشگاه تهران & ۱۴۰۳
\end{tabular*}


\end{document}
